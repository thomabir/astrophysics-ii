\chapter{Nonlinear Perturbations}

\section{Approaches for non-linear structure formation}

Remember that at late times, perturbations become non-linear, and that small-scale perturbations become non-linear first.
Perturbations collapse to form dark halos, filaments, and pancakes.
This structure is called the cosmic web, which is a signature of non-linear gravitational instability in an expanding universe.
% movie (illustris collaboration)

Studying the evolution of non-linear perturbations is very difficult, and no exact analytic solutions have yet been found in the fully non-linear regime.
There are different approaches to tackle the problem:
\begin{itemize}
	\item Numerical simulations
	\item Higher order perturbation theory is only a valid approach for mildly non-linear perturbations and scales.
	On small scales, this does not work.
	\item The Halo model, which is a simplified model, is very successful in describing the formation, evolution, and statistics of collapsed structures.
\end{itemize}

We will study mainly the Halo model. We start with a model that only contains dark matter, and later introduce baryons.

\section{Spherical Collapse}

We consider a spherical top-hat overdensity $\delta$ in an otherwise homogeneous expanding universe, such as in \cref{fig:top-hat}

% TODO top hat
% \begin{marginfigure}
% 	\centering
% 	\includegraphics[width=0.7\textwidth]{img/ch-04/top-hat.pdf}
% 	\caption{A spherical top-hat overdensity. Outside the sphere, the density is $\bar{\rho}$, while inside it is $\rho = \bar{\rho}(1+\delta)$}
% 	\label{fig:top-hat}
% \end{marginfigure}

We make a few simplifying assumptions:
\begin{itemize}
	\item The universe is flat and matter-dominated, such that $\Omega_0 = \Omega_{\text{m}, 0} = 1$.
	\item There is only collisionless dark matter.
\end{itemize}
We now look at a thin shell with radius $r$ inside the top-hat sphere. Newton's second law states
\begin{align*}
	\dv[2]{r}{t}
	&= - \frac{G M}{r},
\end{align*}
where $M$ is the mass inside the radius $r$,
\begin{align*}
	M = M(r) = \frac{4\pi}{3} r^3 \bar{\rho}(1+\delta).
\end{align*}
Note that $M(r)$ is constant as long as the different shells don't cross each other.
The specific energy, or energy per unit mass, of the shell is
\begin{align*}
	\epsilon = \frac{1}{2} \left( \dv{r}{t} \right)^2
	- \frac{G M}{r},
\end{align*}
which is constant because of conservation of energy.
If $\epsilon \geq 0$, the shell expands forever, and if $\epsilon < 0$, the shell might expand for a while, but will eventually contract and collapse. We will focus on the latter case.

The solution of the equation of motion for $\epsilon < 0$ is
\begin{align*}
	r &= A(1- \cos(\theta)) &
	t &= B(\theta - \sin\theta) &
	A &= \frac{GM}{2 \abs{\epsilon}} &
	B &= \frac{GM}{(2 \abs{\epsilon})^{2/3}}
\end{align*}
This is plotted in \cref{fig:collapse-solution}.
The shell expands, reaches a maximum radius $r_\text{ta} = 2 A$ at $t_\text{ta} = \pi B$, and collapses at $t_\text{coll} = 2 t_\text{ta}$.
However, the model will be invalid shortly before $t_\text{coll}$, since shells start crossing and virialize, which will be discussed later.
% TODO 
% \begin{figure}
% 	\centering
% 	\includegraphics[width=0.7\textwidth]{img/ch-04/collapse-solution.pdf}
% 	\caption{The solution of the equation of motion for a collapsing top-hat model.}
% 	\label{fig:collapse-solution}
% \end{figure}



We now set the initial conditions.
At an early time $t_i$, we have $r=r_i$ and $v = v_i$. We assume that at $t_i$, the shell follows the Hubble flow:
\begin{align*}
	v_i
	&= \dv{(a x_i)}{t} 
	= \dot{a} x_i &&\text{no peculiar velocity}\\
	&= H r_i.
\end{align*}
At $t=t_i$,
\begin{align*}
	M
	&= \frac{4\pi}{3} r_i^3 \bar{\rho}(t_i)(1+\delta_i)\\
	\epsilon
	&= \frac{1}{2} v_i^2 - \frac{GM}{r_i}.
\end{align*}
Since we assumed a matter-dominated universe, $a \propto t^{2/3}$, and thus $H = \frac{2}{3} t^{-1}$. We also assumed a flat universe, so
\begin{align*}
	\bar{\rho}
	= \rho_\text{crit}
	= \frac{1}{6\pi G t^2}.
\end{align*}
We can plug these expressions into the definitions of $A$ and $B$ to get
\begin{align*}
	A = \frac{3}{10} \frac{r_i}{\delta_i},
	\qquad
	B = \frac{9}{20} \frac{t_i}{\delta_i}.
\end{align*}

The density inside the top hat is
\begin{align*}
	\rho
	&= \frac{M}{\frac{4\pi}{3}r^3}
	= \frac{3M}{4\pi A^3} (1-\cos\theta)-3
	&&\text{because } r = A(1-\cos\theta),
\end{align*}
while outside it is
\begin{align*}
	\bar{\rho}
	&= \frac{1}{6\pi G t^2}
	= \frac{1}{6 \pi G B^2} (\theta-\sin\theta)^{-2}
	&& \text{because } t = B(\theta - \sin\theta).
\end{align*}
It follows that
\begin{align*}
	1 + \delta
	= \frac{\rho}{\bar{\rho}}
	= \frac{9}{2} \frac{(\theta-\sin\theta)^{2}}{(1-\cos\theta)^3}.
\end{align*}

At early times, $t << t_\text{ta}$, we perform a Taylor expansion in time and $\theta$ to get
\begin{align*}
	\delta \approx \frac{3}{20} \theta^2
	\qquad
	t \approx \frac{B}{6} \theta^3.
\end{align*}
If we plug this into our previous result, we get
\begin{align*}
	\delta
	= \frac{3}{20} \left( \frac{6 t}{B} \right)^{2/3}
	= \frac{3}{20} (6\pi)^{2/3} \left( \frac{t}{t_\text{ta}} \right)^{2/3},
\end{align*}
which is in agreement with what we found in linear perturbation theory for a matter-dominated universe, where we saw $D(z) \propto a \propto t^{2/3}$.
We define
\begin{align*}
	\delta_\text{lin} = \frac{3}{20} (6\pi)^{2/3} \left( \frac{t}{t_\text{ta}} \right)^{2/3},
\end{align*}
and expect $\delta = \delta_\text{lin}$ at early times, but not at later times.

% TODO figure
The green curve shows the density outside the top-hat sphere, where the density decreases as the universe expands.
The blue curve is the linear model, which we have just calculated.
The red curve shows the realistic behaviour.

At turn around,
\begin{align*}
	1 + \delta(t_\text{ta})
	&= \frac{9}{2} \frac{\pi^2}{2^{3}},
\end{align*}
so $\delta(t_\text{ta}) \approx 4.55$. Also,
\begin{align*}
	\delta_\text{lin}(t_\text{ta})
	= \frac{3}{20} (6\pi)^{2/3}
	\approx 1.062.
\end{align*}
At collapse time, we find similarly $\delta(t_\text{coll}) \to \infty$ and $\delta_\text{lin}(t_\text{coll}) \approx 1.686$.
% TODO put the results in a table

For an OCDM model, one can show that 
\begin{align*}
	\delta_c \approx 1.686 [\Omega_m(t_\text{coll})]^{0.0185},
\end{align*}
and for %TODO
\begin{align*}
	\delta_c \approx 1.686 [\Omega_m(t_\text{coll})]^{0.0055}.
\end{align*}






























