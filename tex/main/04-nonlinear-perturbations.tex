\chapter{Nonlinear Perturbations}

\section{Approaches for non-linear structure formation}

Remember that at late times, perturbations become non-linear, and that small-scale perturbations become non-linear first.
Perturbations collapse to form dark halos, filaments, and pancakes.
This structure is called the cosmic web, which is a signature of non-linear gravitational instability in an expanding universe.
% movie (illustris collaboration)

Studying the evolution of non-linear perturbations is very difficult, and no exact analytic solutions have yet been found in the fully non-linear regime.
There are different approaches to tackle the problem:
\begin{itemize}
	\item Numerical simulations
	\item Higher order perturbation theory is only a valid approach for mildly non-linear perturbations and scales.
	On small scales, this does not work.
	\item The Halo model, which is a simplified model, is very successful in describing the formation, evolution, and statistics of collapsed structures.
\end{itemize}

We will study mainly the Halo model. We start with a model that only contains dark matter, and later introduce baryons.

\section{Spherical Collapse}

We consider a spherical overdensity with in an otherwise homogeneous expanding universe. Outside the sphere, the density is $\bar{\rho}$, while inside it is $\rho = \bar{\rho}(1+\delta)$.

We make a few simplifying assumptions:
\begin{itemize}
	\item The universe is flat and matter-dominated, such that $\Omega_0 = \Omega_{\text{m}, 0} = 1$.
	\item There is only collisionless dark matter.
\end{itemize}
We now look at a thin shell with radius $r$ inside the top-hat sphere, and write down its equation of motion. Newton's second law states
\begin{align*}
	\dv[2]{r}{t}
	&= - \frac{G M}{r},
\end{align*}
where $M$ is the mass inside the radius $r$,
\begin{align*}
	M = M(r) = \frac{4\pi r^3}{3} \bar{\rho}(1+\delta).
\end{align*}
Note that $M(r)$ is constant as long as the different shells don't cross each other, which we assume for now.
The energy per unit mass of the shell, also called its specific energy $\epsilon$, is the sum of a kinetic and a potential term:
\begin{align*}
	\epsilon = \frac{1}{2} \left( \dv{r}{t} \right)^2
	- \frac{G M}{r},
\end{align*}
which is constant because of conservation of energy.
If $\epsilon \geq 0$, the shell expands forever, and if $\epsilon < 0$, the shell might expand for a while, but will eventually contract and collapse. We will focus on the latter case.

The solution of the equation of motion for $\epsilon < 0$ is
\begin{align*}
	r &= A(1- \cos \theta) &
	t &= B(\theta - \sin\theta) &
	A &= \frac{GM}{2 \abs{\epsilon}} &
	B &= \frac{GM}{(2 \abs{\epsilon})^{2/3}}
\end{align*}
The solution $r(t)$ is plotted in \cref{fig:shell-collapse-r}.
The shell expands, reaches a maximum radius $r_\text{ta} = 2 A$ at $t_\text{ta} = \pi B$, and collapses at $t_\text{coll} = 2 t_\text{ta}$.
However, the model will be invalid shortly before $t_\text{coll}$, since shells start crossing and virialize, which will be discussed later.
\begin{figure}
	\centering
	\import{img/ch-04/}{shell-evolution-r.pdf_tex}
	\caption{The evolution of the shell radius of a spherical overdensity, with negative total energy. The shell expands, reaches a maxium radius, and collapses, which is described by the analytical solution derived in the text. Shortly before collapse, the solution becomes invalid as shells start crossing.}
	\label{fig:shell-collapse-r}
\end{figure}



We would also like to derive an expression for $\rho(t)$ inside the sphere, and find the critical density which is required for collapse. To do this, we use a few more of our assumptions and perform some approximations.

At an early time $t_i$, we define $r=r_i$ and $v = v_i$. We assume that at $t_i$, the shell follows the Hubble flow:
\begin{align*}
	v_i
	&= \dv{(a x_i)}{t} 
	= \dot{a} x_i &&\text{no peculiar velocity, and } r = a x\\
	&= H r_i.
	&& H = \dot{a}/a 
\end{align*}
Plugging in at $t=t_i$, we find
\begin{align*}
	M
	= \frac{4\pi r_i^3}{3}  \bar{\rho}(t_i)(1+\delta_i),
	\qquad
	\epsilon
	= \frac{v_i^2}{2}  - \frac{GM}{r_i}.
\end{align*}
Since we assumed a matter-dominated universe, $a \propto t^{2/3}$, and $H = 2/(3t)$. We also assumed a flat universe, so
\begin{align*}
	\bar{\rho}
	= \rho_\text{crit}
	= \frac{1}{6\pi G t^2}.
\end{align*}
We can plug these expressions into the definitions of $A$ and $B$ to get
\begin{align*}
	A = \frac{3}{10} \frac{r_i}{\delta_i},
	\qquad
	B = \frac{9}{20} \frac{t_i}{\delta_i}.
\end{align*}

The density inside the top hat sphere is then
\begin{align*}
	\rho
	&= \frac{M}{\frac{4\pi}{3}r^3}
	= \frac{3M}{4\pi A^3} (1-\cos\theta)-3
	&&\text{because } r = A(1-\cos\theta),
\end{align*}
while outside it is
\begin{align*}
	\bar{\rho}
	&= \frac{1}{6\pi G t^2}
	= \frac{1}{6 \pi G B^2} (\theta-\sin\theta)^{-2}
	&& \text{because } t = B(\theta - \sin\theta).
\end{align*}
It follows that
\begin{align*}
	1 + \delta
	= \frac{\rho}{\bar{\rho}}
	= \frac{9}{2} \frac{(\theta-\sin\theta)^{2}}{(1-\cos\theta)^3}.
\end{align*}
Since we want to find $\rho(t)$ instead of $\rho(\theta)$, we have a little more work to do.
An analytical solution is nowhere to be seen, so we try to find at least an approximation.
At early times, $t << t_\text{ta}$, we perform a Taylor expansion in $t$ and $\theta$ to get
\begin{align*}
	\delta \approx \frac{3}{20} \theta^2
	\qquad
	t \approx \frac{B}{6} \theta^3.
\end{align*}
These two expressions can be combined to find
\begin{align*}
	\delta
	= \frac{3}{20} \left( \frac{6 t}{B} \right)^{2/3}
	= \frac{3}{20} (6\pi)^{2/3} \left( \frac{t}{t_\text{ta}} \right)^{2/3},
\end{align*}
which is fortunately\sidenote{Since we assumed early times, linear perturbation theory should still be valid, so comparing our new result to what we found in the previous chapter gives us some reassurance that everything still works correctly.} in agreement with what we found in linear perturbation theory for a matter-dominated universe, where we derived $D(z) \propto a \propto t^{2/3}$.
Note that this result is only valid at early times, and the actual value of $\delta$ might be quite different at later times. To make things clearer, we define
\begin{align*}
	\delta_\text{lin} = \frac{3}{20} (6\pi)^{2/3} \left( \frac{t}{t_\text{ta}} \right)^{2/3},
\end{align*}
and expect $\delta = \delta_\text{lin}$ at early times, but not at late times. We can find $\rho$ from $\delta$, and $\rho_\text{lin}$ from $\delta_\text{lin}$, which is a more intuitive value. The relationship between $\rho$, $\rho_\text{lin}$, and the density outside the sphere, $\bar{\rho}$, is shown in \cref{fig:shell-collapse-rho}.

\begin{figure}
	\centering
	\import{img/ch-04/}{shell-evolution-rho.pdf_tex}
	\caption{The evolution of the density of a spherical overdensity with negative total energy. The exact model is $\rho$, which agrees well with the linearised approximation $\rho_\text{lin}$ at early times, but only until the turn-around time. For reference, the average density $\bar{\rho}$ of the universe outside the sphere is also shown.}
	\label{fig:shell-collapse-rho}
\end{figure}

The overdensity from the linearized model and the exact model at different times are compared in \cref{tab:exact-vs-lin}.
We notice that the overdensities are larger than one, which in hindsight clarifies why we require non-linear perturbation theory. In the previous chapter, we always assumed $\delta \ll 1$, which is obviously not valid any more.
\begin{margintable}
	\begin{tabular}{lll}
		\toprule
		model & $t_\text{ta}$ & $t_\text{coll}$\\
		\midrule
		$\delta$ & 4.55 & $\infty$\\
		$\delta_\text{lin}$ & 1.062 & 1.686\\
		\bottomrule
	\end{tabular}
	\caption{Overdensities in the linearized and the exact model at different times.}
	\label{tab:exact-vs-lin}
\end{margintable}

Depending on the cosmological models, the overdensity $\delta_c$ required for collapse is different. We have so far worked with a simplified CDM model (sCDM), where we assumed that the universe is flat and dominated by collisionless dark matter. For other cosmologies, such as OCDM and ΛCDM, it turns out that multiplying the sCDM result with an additional factor yields a good approximation:
\begin{align*}
	\delta_c \approx
	\begin{cases}
	1.686 & \text{for sCDM}\\
	1.686 [\Omega_m(t_\text{coll})]^{0.0185} & \text{for OCDM}\\
	1.686 [\Omega_m(t_\text{coll})]^{0.0055} & \text{for ΛCDM}
	\end{cases}
\end{align*}






























