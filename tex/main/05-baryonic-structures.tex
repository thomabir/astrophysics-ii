\chapter{Baryonic Structures}


\section{State and distribution of baryons}


We want to include baryons in the study of non-linear structure formation.

% TODO figure big bang againg

\paragraph{Evolution}
Before recombination and decoupling, the baryons were an ionized plasma.

After recombination, they are mostly neutral and in atomic form. At that point, there is little emission, which is why this period is called the Dark Ages.

At later times ($z \approx 6$ -- $20$), the first stars start to form, which emit UV photons that re-ionize the baryons and end the Dark Ages. This is called the Epoch of Reionization (EoR).

Following this, the baryons continue to participate in structure formation, driven by dark matter.


\paragraph{Phases}
\begin{itemize}
	\item Gas
	\begin{itemize}
		\item Hot gas (ionized)
		\item Cold gas (atomic or molecular form)
	\end{itemize}
	\item Stars
\end{itemize}
These different phases interact and convert to each other.


\paragraph{Distribution}
\begin{itemize}
	\item within galaxies: Interstellar medium (ISM)
	\item between galaxies: Intergalactic medium (IGM)
	\item within galaxies clusters: Intra-cluster medium (ICM)
\end{itemize}



\paragraph{Dynamics}
We know that there is five times as much dark matter mass as there is baryonic mass.
As a result, dark matter mostly drives the dynamics of structure formation and provides the backbone for the dynamics of the baryons.
Unlike dark matter, baryons are collisional, and their dynamics include various effects:
\begin{itemize}
	\item Thermodynamical effects: Heating and cooling
	\item Radiative transfer: Interaction between baryons and photons
	\item Hydrodynamical effects: ex. Shocks
	\item Transitions between different phases: ex. star formation
	\item Astrophysical effect: ex. supernova explosions, active galactic nuclei
\end{itemize}
The dynamics of baryons is more complicated than that of dark matter.

% TODO illustris simulation

Generally, some baryons tend to sink to the bottom of potential wells, driven by dark matter on small scales, because baryons have dissipation.

Our approach here is to first review astrophyiscal facts about baryonic structures and discuss approaches for modelling the baryons.





\section{Stars}

Stars are formed by the collapse of cold gas in molecular clouds in the ISM.
They are supported against gravity by nuclear fusion reactions at their centre.
The energy that is produced is transported outwards by photons to the star's outer shells.

\subsection{Sun}
We have already seen that $M_\sol \approx \SI{2e30}{\kg}$.

\subsection{Composition}
The solar composition is shown in TODO.
% Pridmorial: 25% He, only traces of heavier elements
Apparently, the abundance of helium and heavier elements in the sun is different from the primordial abundance.
This is because heavier elements are produced by nuclear reactions in stars.
To characterize these abundances, we define the \emph{metallicity} $z$, which is the mass faction of elements heavier than helium.
For the sun, $z_\sol \approx \SI{2}{\percent}$.


Stellar types are classified based on their surface temperature, which is shown in TODO.
The spectra of different types are shown in TODO.

\subsection{Hertzsprung-Russel diagram}

TODO figure.

Stars are born on the main sequence (MS).
The phenomenological \emph{initial mass function} (IMF) is the distribution of newly born stars of the MS.
Eventually, they have burned all their hydrogen and start moving away from the MS towards the red giant branch (RGB).
Stars with higher masses exhaust their hydrogen first.
As a reg giant runs out of nuclear fuel, they collapse into compact objects. Depending on their mass, this compact object can be a white dwarf, a neutron star, or a black hole.
In the process, supernova explosions can be generated.
They inject kinetic energy into the ISM and disperse the heavy elements that the star generated.
This process is called enrichment of the ISM.

\subsection{Galaxies}
Galaxies contain many stars, so they can be used to track the stellar evolution.
Because the massive blue stars burn their fuel first, new galaxies are blue, and old galaxies are red.










